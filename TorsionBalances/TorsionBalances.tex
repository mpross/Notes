\documentclass{article}

\usepackage{amsmath}
\usepackage{amsfonts}
\usepackage{graphicx}

\begin{document}

\title{Torsion Balances}
\author{M.P.Ross for the E\"ot-Wash Group}

\maketitle

\tableofcontents

\pagebreak

\section{History}
\section{Introduction}
\subsection{Simple Torsion Balance}\label{simple}

\quad A torsion balance, in it's simplest incarnation, is just an extended body, called the "pendulum," suspended from a thin wire, the "torsion fiber." This forms a rotational spring-mass system which has two intrinsic parameter (ignoring loss terms): the moment of inertia, $I$, and the torsional spring constant, $\kappa$. The primary degree of freedom of this system is rotation of the pendulum around the axis of the fiber which we call torsion. See Section~\ref{swing} for discussion of other degrees of freedom.

Due to historical reasons, the classical example of a torsion balance is a dumb-bell shaped pendulum suspended from a thin metal fiber, shown in Figure~\ref{simpleFig}. The pendulum is formed by a massless rod with two equal mass "test masses" attached to each end. The torsion fiber is then attached to the rod at equal distance to each test mass. This provides a prototypical model of a torsion balance which we will analyze in detail. 

\begin{figure}[!h]
\begin{centering}
\includegraphics[width=0.65\textwidth]{SimpleTorsionBalance.pdf}
\caption{A simple torsion balance system.}\label{simpleFig}
\end{centering}
\end{figure}

Modern torsional balance apparatus typically have pendulums with more complex geometry and may have multiple suspension stages. 

\subsection{Equation of Motion}

As mentioned in Section~\ref{simple}, the simple torsion balance is described with two primary parameter: the moment of inertia, $I$, which is determined by the pendulum geometry, and the torsional spring constant, $\kappa$, which is determined by the torsion fiber size and material. 

\section{Mechanics}
\subsection{Torque Sensing}
\subsection{Inertial Sensing}
\subsection{Fiber Selection}
\subsection{Pendulum Design}

\section{Complications}
\subsection{Swing Modes} \label{swing}
\subsection{Centrifugal Force}
\subsection{Multiple Pendulums}

\section{Noise Sources and Mitigation}
\subsection{Thermal Noise}
\subsection{Seismic Motion}
\subsection{Electrostatic Couplings}
\subsection{Magnetic Noise}
\subsection{Gas Damping}
\subsection{Gravity Gradients}

\section{Case Study Experiments}
\subsection{Inverse Square Law}
\subsection{Equivalence Principle}
\subsection{Gravitational Wave Detection}
\end{document}
